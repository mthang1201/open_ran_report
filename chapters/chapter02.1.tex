\section{So sánh kiến trúc RAN truyền thống và O-RAN}
Các mạng di động truyền thống trong lịch sử chủ yếu áp dụng kiến trúc RAN phân tán (Distributed RAN - D-RAN), trong đó mỗi trạm gốc tích hợp cả chức năng xử lý băng tần gốc (baseband) và chức năng vô tuyến. Mô hình này phù hợp với các thế hệ di động đầu tiên (2G/3G/4G), nhưng tồn tại nhiều hạn chế về khả năng mở rộng, tính tương tác giữa các nhà cung cấp và khả năng quản lý tập trung. Với sự ra đời của 5G và tương lai là 6G, ngành công nghiệp viễn thông đang chuyển dịch sang các giải pháp linh hoạt và định nghĩa bằng phần mềm. Open RAN (O-RAN) đưa ra một kiến trúc mở, mô-đun và tách rời bằng cách phân tách các chức năng RAN thành ba thành phần riêng biệt: CU, DU và RU, cùng với các giao diện tiêu chuẩn hóa \cite{oran-wg6-architecture}. Bài viết này phân tích các điểm khác biệt về kiến trúc và những hàm ý giữa D-RAN và O-RAN.

Trong D-RAN, mỗi trạm thu phát (cell site) chứa đồng thời một Bộ xử lý băng tần gốc (BBU) và một Bộ vô tuyến từ xa (RRU). BBU thực hiện toàn bộ xử lý tầng baseband, bao gồm PHY, MAC, RLC và các tầng giao thức cao hơn, trong khi RRU đảm nhận các chức năng đầu vào/ra RF tương tự. Sự tích hợp này giúp đơn giản hóa việc đồng bộ hóa và giảm độ trễ đường truyền fronthaul, nhưng lại dẫn đến hệ thống cứng nhắc và phụ thuộc vào nhà cung cấp. Do phần cứng và phần mềm gắn kết chặt chẽ, việc mở rộng hoặc nâng cấp hệ thống thường đòi hỏi phải thay thế toàn bộ các thành phần độc quyền. Ngoài ra, khả năng tối ưu hóa mạng và chia sẻ tài nguyên giữa các trạm bị hạn chế do xử lý vẫn diễn ra độc lập tại từng vị trí \cite{3gpp-tr38.801}.

O-RAN khác biệt với cách tiếp cận nguyên khối của D-RAN bằng việc phân tách chức năng rõ ràng giữa RU, DU và CU. RU đảm nhận xử lý PHY thấp và RF, DU chịu trách nhiệm xử lý PHY cao, MAC và RLC, trong khi CU điều khiển các tầng SDAP, PDCP và RRC. Các thành phần này giao tiếp thông qua các giao diện mở: giao diện F1 giữa CU và DU, và giao diện fronthaul mở (thường là chia tách 7.2x) giữa DU và RU \cite{khurshid2024oran}. Tính mô-đun này cho phép các thiết bị từ nhiều nhà cung cấp khác nhau tương tác, đồng thời hỗ trợ ảo hóa các chức năng CU/DU trên phần cứng thương mại thông dụng \cite{ericsson-openran}.

Một điểm khác biệt lớn giữa D-RAN và O-RAN nằm ở tính linh hoạt khi triển khai. Trong khi D-RAN đồng vị tất cả xử lý tại trạm thu phát, thì O-RAN hỗ trợ việc tập trung hóa CU tại các trung tâm dữ liệu khu vực và phân phối DU gần biên mạng. Sự phân tách này cho phép phân bổ tài nguyên động, giảm chi phí đầu tư (CAPEX) và nâng cao khả năng mở rộng. Hơn nữa, O-RAN còn hỗ trợ các chức năng tiên tiến như bộ điều khiển RAN thông minh (RIC), cho phép tối ưu hóa dựa trên AI/ML gần như theo thời gian thực \cite{oran-wg6-architecture}.

Tuy có nhiều ưu điểm, O-RAN cũng đặt ra thách thức trong việc tích hợp do môi trường đa nhà cung cấp và yêu cầu nghiêm ngặt về đường truyền fronthaul. Các liên kết có tốc độ cao và độ trễ thấp là điều kiện bắt buộc để đảm bảo hiệu năng trong kiến trúc chia tách. Ngược lại, D-RAN đơn giản hơn trong triển khai và đã được kiểm chứng về độ tin cậy, nhưng thiếu sự linh hoạt và tính mở cần thiết cho các trường hợp sử dụng mới như chia sẻ mạng (network slicing) và mạng 5G riêng (private 5G) \cite{ericsson-openran}.

Kiến trúc D-RAN truyền thống đã đóng vai trò là nền tảng của mạng di động trong nhiều thập kỷ, mang lại sự đơn giản và độ ổn định cho các thế hệ mạng di động trước. Tuy nhiên, nhu cầu ngày càng tăng về tính linh hoạt, hiệu quả chi phí và phân biệt dịch vụ đang thúc đẩy quá trình chuyển đổi sang hệ thống mở và tách rời. O-RAN, với việc phân tách chức năng và các giao diện tiêu chuẩn, đại diện cho một sự chuyển đổi mô hình hướng tới giải pháp RAN gốc đám mây và trung lập với nhà cung cấp. Mặc dù vẫn còn tồn tại các thách thức về tích hợp và tối ưu hóa hiệu năng, nhưng những lợi ích tiềm năng của O-RAN đang định vị nó như một yếu tố then chốt trong việc phát triển mạng di động tương lai \cite{khurshid2024oran}.