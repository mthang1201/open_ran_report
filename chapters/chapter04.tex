Báo cáo đã trình bày chi tiết các vấn đề cốt lõi liên quan đến kiến trúc mạng truy cập vô tuyến mở (Open RAN), với việc tập trung vào vai trò của các thành phần quan trọng như RIC và các giao diện mở (E2, A1, O1). Thông qua nghiên cứu, rõ ràng thấy được các lợi ích nổi bật của Open RAN so với các kiến trúc mạng truyền thống, đặc biệt là khả năng linh hoạt, tương tác đa nhà cung cấp và triển khai các ứng dụng AI và DRL thông minh trong việc quản lý tài nguyên mạng.

Phần thực nghiệm đã được thực hiện bằng việc sử dụng bộ công cụ mô phỏng RIMEDO-TS, thể hiện rõ tính ứng dụng thực tiễn của kiến trúc Open RAN và công nghệ điều khiển thông minh DRL. Kết quả thực nghiệm chứng minh rằng việc triển khai các xApp thông minh dựa trên thuật toán DQN mang lại hiệu quả cao trong việc cân bằng tải, tối ưu hóa tài nguyên và nâng cao hiệu suất tổng thể của mạng.

Kết quả nghiên cứu và thực nghiệm khẳng định rằng kiến trúc Open RAN cùng với công nghệ trí tuệ nhân tạo và học tăng cường sâu có tiềm năng lớn trong việc tái định hình và nâng cấp mạng viễn thông hiện đại. Việc tiếp tục nghiên cứu sâu hơn và mở rộng các ứng dụng này trong tương lai sẽ giúp tối ưu hóa đáng kể hiệu năng và chi phí vận hành của các nhà cung cấp dịch vụ viễn thông, góp phần quan trọng vào sự phát triển của ngành công nghiệp viễn thông nói chung.

